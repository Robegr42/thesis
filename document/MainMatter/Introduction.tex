\chapter*{Introducción}\label{chapter:introduction}
\addcontentsline{toc}{chapter}{Introducción}

En los últimos años se ha hecho creciente el uso, investigación y aplicación
del Aprendizaje Automático, en inglés \emph{Machine Learning}(ML). Sin embargo,
el uso de estás técnicas está sujeto a conocimiento de mucha teoría
matemático-computacional, lo cual constituye una barrera para nuevos usuarios.
Por ejemplo, para científicos de esta rama, crear una aplicación de ML está
sujeto a la selección de disímiles tipos de algoritmos(redes neuronales,
modelos bayesianos, algoritmos de \emph{clustering}, etc) y, a la vez, a
ajustar los númerosos hiperparámetros del algoritmo selecionado. Incluso en la
mayoría de los casos se invierte una cantidad de enorme de tiempo y recursos en
la creación de un modelo que se ajuste a los requerimientos del problema a
causa del proceso de prueba y error que es repetido en cada aplicación para
desarrollar modelos eficientes.

Como consecuencia del costo de empleo de algoritmos y técnicas de ML, surge una
nueva idea para automatizar el proceso de ML. El Aprendizaje de Máquina
Automático, en inglés \emph{Automated Machine Learning}(AutoML). AutoML ha
facilitado el proceso de implementación, despliegue, modelación y desarrollo de
algoritmos de ML.

El surgimiento de AutoML ha reducido la carga de trabajo de los científicos de
datos y ha permitido un mayor consumo de aplicaciones, reduciendo el tiempo de
ejecución de procesos, ayudando, además, a no expertos en el ámbito de la
computación. Por lo tanto, AutoML hace accesible enfoques de aprendizaje
automático a los usuarios no expertos que están interesados en aplicarlos,
pero no tienen los recursos para aprender sobre las tecnologías involucradas
en detalle o simplemente necesitan de estos recursos pero son de una rama
distinta de la ciencia. Con el crecimiento exponencial del poder computacional
y de los datos digitales, AutoML se ha convertido en un tema de creciente
importancia tanto en la industria como en la academia.

Aún así existe la incapacidad de poder reusar conocimiento previo que ya fue
procesado. Una de las limitaciones presentes en los primeros sistemas de AutoML
consiste en su inhabilidad de reusar conocimiento previo para solucionar nuevas
tareas. Para cerrar esta brecha, las herramientas de AutoML comenzaron a
aplicar técnicas de meta-learning, las cuales tienen el objetivo de obtener
modelos para nuevas tareas usando experiencias previas. Meta-learning, o
\emph{aprender a aprender}, es la ciencia de observar sistemáticamente cómo se
desempeñan los diferentes enfoques de aprendizaje automático en una amplia
gama de tareas de aprendizaje, y luego aprender de esta experiencia, o
meta-datos, para aprender nuevas tareas mucho más rápido de lo que sería
posible de otra manera. Esto no solo acelera y mejora drásticamente el diseño
de algoritmos de aprendizaje automático, sino que también nos permite
reemplazar algoritmos diseñados a mano con enfoques novedosos aprendidos de una
manera basada en datos. Este tipo de estrategias ayudan a disminuir el costo de
aplicar AutoML, al relacionar un nuevo conjunto de datos con los mejores flujos
obtenidos en problemas similares previamente resueltos.\\

\textbf{\Large Problema}\\

A pesar de la creciente cantidad de herramientas de AutoML desarrolladas en los
últimos años, la generación automática de soluciones de ML es
computacionalmente costosa y toma mucho tiempo. Las razones de estos defectos
incluyen un espacio de búsqueda muy grande, tanto para algoritmos simples de ML
como para otras arquitecturas más complejas como redes neuronales, y que la
evaluación de incluso un solo algoritmo en un dataset grande puede requerir
horas. Esto hace que el proceso de desarrollo de modelos de ML siga siendo un
proceso lento, limitando la aplicabilidad de AutoML a problemas prácticos en la
industria, así como su potencial para acelerar la investigación científica.\\

\textbf{\Large Motivación}\\

En esta tesis se realiza una propuesta para superar estos obstáculos. El
principal objetivo del enfoque propuesto es asistir a los profesionales de la
industria, a los investigadores en ciencias de datos y de otras ramas de la
ciencia en la selección de modelos incorporando un componente en los
sistemas AutoML, los cuales aumentarán el rendimiento de los mismos, se
acelerarán los procesos de búsqueda proveyendo un conjunto inicial de
algoritmos y las configuraciones de sus hiperparámetros.

Mediante meta-learning, se intenta ganar perspectiva sacada de los meta-datos
de experimentos de aprendizaje automático. Los resultados de cada entrenamiento
son guardados con caracterizaciones del dataset y sus detalles de rendimiento,
y son usados en las ejecuciones futuras.

Sin embargo, estas herramientas de meta-learning no son suficientemente
flexibles para ser utilizadas en problemas prácticos que requieren la
combinación de algoritmos y tecnologías de diferente naturaleza. Las técnicas
actuales de meta-learning se centran principalmente en un subconjunto
específico de algoritmos, a menudo adaptados a una biblioteca o conjunto de
herramientas.\\

\textbf{\Large Antecedentes}\\

Esta tesis forma parte de las líneas de investigación del grupo de Inteligencia
Artificial de la Facultad de Matemática y Computación de la Universidad de La
Habana. En dicho grupo se ha diseñado el sistema de AutoGOAL, por lo que esta
herramienta es la usada para la incorporación de conocimiento experto mediante
meta-learning. AutoGOAL es un sistema AutoML implementado como una biblioteca
de código abierto en el lenguaje de programación Python. Utiliza técnicas
heterogéneas, que a diferencia de otros sistemas, puede construir
automáticamente flujos de aprendizaje automático que combinen técnicas y
algoritmos de diferentes bibliotecas, incluidos clasificadores lineales,
herramientas de procesamiento de lenguaje natural y redes neuronales.\\

\textbf{\Large Objetivos}\\

El objetivo general de esta tesis es el diseño de una estrategia de
meta-learning para métodos genéricos de AutoML. La estrategia implementada
tendrá el objetivo de acelerar el proceso de búsqueda de AutoML añadiendo
conocimiento previo, de tal manera que se obtengan mejores resultados en el
mismo período de tiempo. Para realizar esto será necesario:

\begin{itemize}
    \item Estudiar diferentes herramientas de meta-learning, así como sistemas
    de Auto-ML presentes en la literatura.
    \item Definir una representación para las soluciones generadas de los
    datasets.
    \item Diseñar un algoritmo para recomendar una lista de soluciones para un
    nuevo dataset basado en estos datos.
    \item Incorporar el conocimiento previo obtenido con meta-learning en
    AutoGOAL para realizar una búsqueda de algoritmos más eficiente.
    
\end{itemize}

\textbf{\Large Estructura de la tesis}\\

El resto de la tesis está organizada de la siguiente manera. El Capítulo
\ref{chapter:state-of-the-art} introduce los problemas de meta-learning y
AutoML y las técnicas que a menudo son aplicadas para tratar con estos
problemas. Esto es seguido por un resumen de los trabajos relacionados con
meta-learning para la selección de algoritmos y de distintas herramientas de
AutoML. Luego, en el Capítulo \ref{chapter:proposal} se describe la estrategia
de meta-learning propuesta para AutoML, incluyendo un análisis de las
meta-características usadas y las estrategias desarrolladas. En el Capítulo
\ref{chapter:implementation} se exponen brevemente los aspectos de la
metodología experimental adoptada, se investigan los resultados obtenidos y se
realiza una discusión de los mismos. Posteriormente, se presentan las
conclusiones finales de esta tesis y algunas recomendaciones para estudios
futuros.
