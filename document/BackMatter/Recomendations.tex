\begin{recomendations}
    El enfoque de meta-learning presentado en esta Tesis es utilizable en
    problemas prácticos y proporciona información importante en el proceso de
    AutoML, lo que le permite obtener mejores resultados en la búsqueda de
    flujos de algoritmos. Sin embargo, aún se encuentra en una etapa de
    desarrollo inicial, por lo que es necesario seguir mejorando sus capacidades
    mediante un estudio más exhaustivo.

    El conocimiento obtenido con la estrategia de meta-learning fue añadido a
    AutoML en el proceso de inicialización de la búsqueda de flujos de
    algoritmos. Sin embargo, se recomienda una experimentación más exhaustiva
    con conjuntos de datos más genéricos, para probar la efectividad de la
    propuesta en problemas reales.

    Además, en la experimentación realizada en esta investigación se analizan
    los resultados obtenidos en el proceso de inicialización de AutoGOAL con un
    conjunto inicial de \emph{k} flujos de algoritmos(en esta propuesta se
    analizan los valores de 10,25 y 50). El tamaño de este conjunto inicial,
    que es el utilizado para añadir conocimiento previo al proceso de
    AutoML, puede variar considerablemente el rendimiento final obtenido en la
    búsqueda de flujos. Por lo tanto, se recomienda la realización
    de experimentos para encontrar el tamaño de \emph{k} óptimo para ser
    añadido como conjunto inicial a AutoML.
\end{recomendations}
