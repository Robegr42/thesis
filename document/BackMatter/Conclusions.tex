\begin{conclusions}
    La inteligencia artificial, y en particular el aprendizaje automático, es
    cada vez más demandado en la industria, debido al potencial que tiene para
    automatizar los procesos más complejos. Las organizaciones están repletas
    de datos, pero carecen de personas con la experiencia técnica necesaria
    para transformar estos datos en conocimientos prácticos
    ~\brackcite{miller2017quant}. Entre las principales dificultades para
    aplicar extensivamente técnicas de aprendizaje automático en problemas
    reales, están la poca disponibilidad de expertos unido al costo de diseñar,
    implementar y evaluar este tipo de soluciones.

    Las organizaciones se están volcando cada vez más hacia la automatización
    en el trabajo de la ciencia de datos, con el objetivo de liberar a los
    expertos de las tareas menos creativas en la implementación de sistemas de
    aprendizaje automático. Por lo tanto, han comenzado con la adopción de
    técnicas que automatizan la creación de modelos de aprendizaje automático
    ~\brackcite{drozdal2020trust, wang2019humanai}. Sin embargo, la adopción de
    esta tecnología en entornos empresariales ha tenido dificultades.
    Actualmente, las herramientas de AutoML tienen restricciones respecto a lo
    que pueden realizar de manera flexible~\brackcite{crisan2021fits}.

    Una de las limitaciones presente en los primeros sistemas de AutoML es su
    inhabilidad de reusar conocimiento previo para solucionar nuevas tareas.
    Para cerrar esta brecha, las herramientas de AutoML comenzaron a aplicar
    técnicas de meta-learning, las cuales tienen el objetivo de obtener modelos
    para nuevas tareas usando experiencias de aprendizaje anteriores. Este
    tipo de estrategias ayudan a disminuir el costo de aplicar AutoML, al
    relacionar un nuevo conjunto de datos con los mejores flujos obtenidos en
    problemas similares previamente resueltos.

    Aunque existen varias herramientas de meta-learning que han sido exitosas
    al aplicarse a AutoML, resolviendo problemas específicos de inteligencia
    artificial, estas herramientas son aún poco flexibles para ser utilizadas
    en problemas prácticos que requieren la combinación de algoritmos y
    tecnologías de diferente naturaleza.

    El enfoque desarrollado se recomienda como un paso preliminar para otras
    soluciones más costosas computacionalmente, como por ejemplo, para la
    inicialización de sistemas de AutoML. En esta investigación AutoGOAL
    también es empleado como herramienta complementaria en el proceso de
    búsqueda de flujos. Por lo tanto, se describe como se realiza la
    incorporación de conocimiento experto a la estrategia de búsqueda utilizada
    por AutoGOAL: Evolución Gramatical Probabilística.

    La propuesta de meta-learning consiste en la selección de un conjunto de
    flujos de algoritmos para ser propuestos en la inicialización de la
    optimización de AutoGOAL. La elección de este conjunto de flujos se realiza
    mediante un enfoque de ranking, en el que para un nuevo conjunto de datos
    se seleccionan los \emph{k} mejores flujos de algoritmos.

\end{conclusions}
