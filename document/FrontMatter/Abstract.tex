\begin{resumen}
	Con el desarrollo de la ciencia en el campo de la Inteligencia Artificial,
	la experimentación y el uso de aplicaciones de Aprendizaje de Máquinas
	Automático(AutoML) ha ido en ascenso en los últimos tiempos, llegando a
	campos y ramas de la ciencia tan diversas como la Biología, Neurociencias y
	Medicina, donde han demostrado ser de gran efectividad. A pesar del
	reciente éxito de AutoML, todavía quedan muchos desafíos El uso de
	Meta-learning proporcionará un nuevo camino en la optimización y
	rendimiento de estos programas aprovechando conocimiento ya computado, de
	esta forma se aprende de tareas previamente analizadas, proporcionando una
	idea de solución a futuros	problemas de características semejantes,
	acelerando el proceso de AutoML y obteniendo mejores resultados en el mismo
	período de tiempo. 

	Como sistema de AutoML se escogió AutoGOAL, que destaca por su capacidad de
	generar soluciones eficaces para una amplia gama de dominios, permitiéndole
	resolver una gran cantidad de tareas. La propuesta de meta-learning
	presentada en esta tesis aborda una gran variedad de tareas mediante la
	selección de características capaces de representar el espacio definido por
	ellas.
\end{resumen}

\begin{abstract}
	Resumen en inglés
\end{abstract}