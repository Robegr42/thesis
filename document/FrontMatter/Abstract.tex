\begin{resumen}
	Con el desarrollo de la ciencia en el campo de la Inteligencia Artificial,
	la experimentación y el uso de aplicaciones de Aprendizaje de Máquinas
	Automático~(AutoML) ha ido en ascenso en los últimos tiempos, llegando a
	campos y ramas de la ciencia tan diversas como la Biología, Neurociencias y
	Medicina, donde han demostrado ser de gran efectividad. A pesar del
	reciente éxito de AutoML, todavía quedan muchos desafíos. El uso de
	Meta-learning proporcionará un nuevo camino en la optimización y
	rendimiento de estos programas aprovechando conocimiento ya computado, de
	esta forma se aprende de tareas previamente analizadas, proporcionando una
	idea de solución a futuros	problemas de características semejantes,
	acelerando el proceso de AutoML y obteniendo mejores resultados en el mismo
	período de tiempo. 

	Como sistema de AutoML se escogió AutoGOAL, que destaca por su capacidad de
	generar soluciones eficaces para una amplia gama de dominios, permitiéndole
	resolver una gran cantidad de tareas. La propuesta de meta-learning
	presentada en esta tesis aborda una gran variedad de tareas mediante la
	selección de características capaces de representar el espacio definido por
	ellas.
\end{resumen}

\begin{abstract}
	With the development of science in the field of Artificial Intelligence,
	experimentation and use of Automated Machine Learning~(AutoML) applications
	has been on the rise in recent times, reaching fields and branches of
	science as diverse as Biology, Neurosciences and Medicine, where they have
	proven to be highly effective. Despite of recent success of AutoML, many
	challenges still remain. The use of Meta-learning will provide a new path
	in optimization and performance of these programs taking advantage of
	knowledge already computed, in this way, we learn from previously analyzed
	tasks, providing an idea of solution to future problems with similar
	characteristics, speeding up the AutoML process and getting better results
	in the same amount of time.
	
	AutoGOAL was chosen as the AutoML system, which stands out for its ability
	to generate effective solutions for a wide range of domains, allowing it
	to solve a large number of tasks. The meta-learning proposal presented in
	this thesis addresses a wide variety of tasks through the selection of
	features capable of representing the space defined by them.
\end{abstract}