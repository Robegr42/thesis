\begin{opinion}
    En la actualidad el Aprendizaje Automático ha llegado a todas las ramas de
    la industria, ayudando a resolver un gran número de problemas pero creando
    la necesidad de un enorme número de expertos para poder utilizar las
    herramientas adecuadas en cada caso.

    En este escenario el AutoML propone una solución ayudando con la selección
    de forma automática de las mejores soluciones con el problema añadido de
    que incrementa el costo computacional ya que tiene que evaluar muchas
    soluciones para resolver cada problema. Realizando esta tarea cada vez.
    El área de investigación en que incursiona el estudiante propone un enfoque
    para que los sistemas de AutoML puedan aprovechar la información de
    evaluaciones anteriores.

    El  estudiante Roberto García Rodriguez en esta investigación se adentra
    en un tema del estado del arte de gran actualidad y para eso tuvo que
    utilizar conocimientos de varias asignaturas de la carrera y otros que no
    son parte del currículum estándar. Su propuesta implicó estudiar el estado
    del arte de las herramientas de AutoML y su uso en la resolución de
    diferentes tareas. Además implicó conocer una herramienta de AutoML nueva
    e incorporar su estrategia para evaluar y comparar sus resultados en la
    práctica.

    Sus resultados resultan muy prometedores,  permitiendo abrir la puerta a
    resolver problemas de metalearning enl AutoML heterogéneo. Esta mejora es
    considerable para una herramienta del estado del arte, que ya lograba
    resultados comparables a las mejores herramientas de AutoML existentes.
    Más aún, las estrategias desarrolladas en esta investigación han sido
    aplicadas solamente a una parte pequeña del proceso de 
    AutoML pero pueden ser extendidos fácilmente.

    Para poder afrontar el trabajo, el estudiante tuvo que revisar literatura
    científica relacionada con la temática así como soluciones existentes y
    bibliotecas de software que pueden ser apropiadas para su utilización.
    Todo ello con sentido crítico, determinando las mejores aproximaciones y
    también las dificultades que presentan.

    Todo el trabajo fue realizado por el estudiante con una elevada constancia,
    capacidad de trabajo y habilidades, tanto de gestión, como de desarrollo y 
    de investigación.

    Por estas razones recomiendo que le sea otorgado al estudiante Roberto
    García Rodríguez el título de Licenciado en Ciencia de la Computación.

    \begingroup
        \centering
        \wildcard{Dr. Suilan Estevez Velarde}
    \endgroup

\end{opinion}